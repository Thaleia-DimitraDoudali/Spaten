\chapter{Background}

\section{Databases}

A database \cite{8} is an organized collection of data, which can be managed in an easy way. A database management system is a computer software application 
that interacts with the user and other applications, in order to create, access, manage and execute queries at the data stored in the database. 
Databases are used in a wide range of services, such as bank, university and communication platforms. 

\subsection{Relational databases}

A relational database is a digital database whose organization is based on the relational model of data \cite{9}. According to this model, data are organized in tables 
with rows and columns. Each table represents a relation variable, each row is an implementation of this variable and each column is an attribute of the variable. 
Tables can be associated with one-to-one, one-to-many and many-to-many relationships. More specifically, a relation is a set of table rows that have common attributes. 
Moreover, each row can be uniquely identified by a certain attribute, called primary key. When a primary key is common between two tables, then it becomes 
a foreign key for the second table. Finally, a database index is a data structure that improves the speed of data retrieval operations on a database table 
at the cost of additional writes and storage space to maintain the index data structure.

\subsection{ACID properties}

ACID (Atomicity, Consistency, Isolation, Durability) \cite{10} is a set of properties that guarantee that database transactions are processed reliably. 
In the context of databases, a single logical operation on the data is called a transaction. 

\subsubsection{Atomicity}

Atomicity requires that each transaction be "all or nothing": 
if one part of the transaction fails, the entire transaction fails, and the database state is left unchanged. An atomic system must guarantee atomicity in each 
and every situation, including power failures, errors, and crashes. To the outside world, a committed transaction appears (by its effects on the database) to be 
indivisible ("atomic"), and an aborted transaction does not happen.

\subsubsection{Consistency}

The consistency property ensures that any transaction will bring the database from one valid state to another. Any data written to the database must be valid 
according to all defined rules, including constraints, cascades, triggers, and any combination thereof. This does not guarantee correctness of the transaction 
in all ways the application programmer might have wanted (that is the responsibility of application-level code) but merely that any programming errors cannot 
result in the violation of any defined rules.

\subsubsection{Isolation}

The isolation property ensures that the concurrent execution of transactions results in a system state that would be obtained if transactions were 
executed serially. Providing isolation is the main goal of concurrency control. Depending on concurrency control method 
(i.e. if it uses strict - as opposed to relaxed - serializability), the effects of an incomplete transaction might not even be visible to another transaction.

\subsubsection{Durability}

Durability means that once a transaction has been committed, it will remain so, even in the event of power loss, crashes, or errors. In a relational database, 
for instance, once a group of SQL statements execute, the results need to be stored permanently (even if the database crashes immediately thereafter). 
To defend against power loss, transactions (or their effects) must be recorded in a non-volatile memory.

\section{PostgreSQL}

PostgreSQL \cite{11} is an open source object-relational database management system. It runs on all major operating systems, including Linux, UNIX (AIX, BSD, HP-UX, 
SGI IRIX, Mac OS X, Solaris, Tru64), and Windows. It is fully ACID compliant, has full support for foreign keys, joins, views, triggers, and stored procedures 
(in multiple languages). It includes most SQL:2008 data types, including INTEGER, NUMERIC, BOOLEAN, CHAR, VARCHAR, DATE, INTERVAL, and TIMESTAMP. It also supports 
storage of binary large objects, including pictures, sounds, or video. Moreover, it has native programming interfaces for C/C++, Java, .Net, Perl, Python, Ruby, Tcl, ODBC. 
It supports international character sets, multibyte character encodings, Unicode, and it is locale-aware for sorting, case-sensitivity, and formatting. 
In addition, it supports unlimited database size, rows and indexes per table, 32TB table size, 1.6TB row size and 1GB field size. PostgreSQL manages 
database access permissions using the concept of roles. A role can be thought of as either a database user, or a group of database users, depending on how 
the role is set up. Roles can own database objects (for example, tables) and can assign privileges on those objects to other roles to control 
who has access to which objects. Furthermore, it is possible to grant membership in a role to another role, thus allowing the member role use of privileges 
assigned to the role it is a member of. Finally, PostgreSQL comes with several extensions that add extra capabilities in its function and usage. 

\subsection{PostGIS}

PostGIS \cite{12} is a spatial database extender for PostgreSQL object-relational database. It adds support for geographic objects allowing location queries 
to be run in SQL. It defines new data types, functions, operators and indexes especially for geographic objects. 

\subsubsection{Geography type}

The geography type provides native support for spatial features represented on geographic coordinates. A specific point on a map can be identified by it's 
geographic coordinates, stated in the form of (latitude, longitude). Geographic coordinates are spherical coordinates expressed in angular units (degrees). 
The basis for the PostGIS geographic type is a sphere. The shortest path between two points on the sphere is a great circle arc. That means that calculations 
on geographies (areas, distances, lengths, intersections, etc) must be calculated on the sphere, using more complicated mathematics. For more accurate 
measurements, the calculations must take the actual spheroidal shape of the world into account. There are several functions and operators 
that take as input or return as output a geography data type object. One very useful one, is the function ST\_DWithin. 
This function returns true if the geographies are within the specified distance of one another. Units are in meters and measurement is defaulted 
to measure around spheroid. 

\subsection{Indexes}

A database index is a data structure that improves the speed of data retrieval operations on a database table at the cost of additional writes and storage 
space to maintain the index data structure. Indexes are used to quickly locate data without having to search every row in a database table every time a database 
table is accessed. Indexes can be created using one or more columns of a database table, providing the basis for both rapid random lookups and efficient access 
of ordered records.

\subsubsection{B-tree}

PostgreSQL uses by default a B-tree index on a table column. B-tree \cite{13} is a tree data structure that keeps data sorted and allows searches, sequential access, 
insertions, and deletions in logarithmic time. The B-tree is a generalization of a binary search tree in that a node can have more than two children.
More specifically, it is a balanced tree whose nodes contain pointers to a table's records and a number of keys. The keys act as separation values which 
divide its subtrees. For example, if an internal node has 3 child nodes (or subtrees) then it must have 2 keys: a1 and a2. All values in the leftmost subtree 
will be less than a1, all values in the middle subtree will be between a1 and a2, and all values in the rightmost subtree will be greater than a2. 
Unlike self-balancing binary search trees, the B-tree is optimized for systems that read and write large blocks of data.

\subsubsection{R-tree}

R-trees \cite{14} are tree data structures used for spatial access methods, such as indexing multi-dimensional information like geographical coordinates. 
The key idea of the data structure is to group nearby objects and represent them with their minimum bounding rectangle in the next higher level of the tree; 
the "R" in R-tree is for rectangle. Since all objects lie within this bounding rectangle, a query that does not intersect the bounding rectangle also cannot 
intersect any of the contained objects. At the leaf level, each rectangle describes a single object; at higher levels the aggregation of an increasing number 
of objects. This can also be seen as an increasingly coarse approximation of the data set. 
R-tree's searching lgorithms use the bounding boxes to decide whether or not to search inside a subtree. In this way, most of the nodes in the tree are never 
read during a search. Like B-trees, this makes R-trees suitable for large data sets and databases, where nodes can be paged to memory when needed, and the whole 
tree cannot be kept in main memory. 

\subsubsection{GiST}

GiST stands for Generalized Search Tree. It is a balanced, tree-structured access method, that acts as a base template in which to implement arbitrary 
indexing schemes. B-trees, R-trees and many other indexing schemes can be implemented in GiST. GiST tree nodes contain a pair of values in the form of (p, prt). 
The first value, p, is used as a search key and the second value, ptr, is a pointer to the data if the node is a leaf or a pointer to another node 
if the node is intermediate. The search key p represents an attribute that becomes true for all data that can be reached through the pointer ptr. 
Also, GiST index can handle any query predicate, as long as certain functions, that influence the behavior of the search keys, are implemented. 

PostGIS can use GiST index at a table attribute. When a table column is a geography data type, then GiST will use an improved version of an R-tree index 
(R-tree-over-GiST scheme). PostgreSQL doesn't allow, at the most recent versions, the use of standard R-tree, because this type of index 
cannot handle attributes with size bigger than 8K and fail when a geography column is null. 















