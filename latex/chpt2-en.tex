\chapter{Background}

\section{Databases}

A database \cite{8} is an organized collection of data, which can be managed in an easy way. A database management system is a computer software application 
that interacts with the user and other applications, in order to create, access, manage and execute queries at the data stored in the database. 
Databases are used in a wide range of services, such as bank, university and communication platforms. 

\subsection{Relational databases}

A relational database is a digital database whose organization is based on the relational model of data \cite{9}. According to this model, data are organized in tables 
with rows and columns. Each table represents a relation variable, each row is an implementation of this variable and each column is an attribute of the variable. 
Tables can be associated with one-to-one, one-to-many and many-to-many relationships. More specifically, a relation is a set of table rows that have common attributes. 
Moreover, each row can be uniquely identified by a certain attribute, called primary key. When a primary key is common between two tables, then it becomes 
a foreign key for the second table. Finally, a database index is a data structure that improves the speed of data retrieval operations on a database table 
at the cost of additional writes and storage space to maintain the index data structure.

\subsection{ACID}

In computer science, ACID (Atomicity, Consistency, Isolation, Durability) \cite{10} is a set of properties that guarantee that database transactions are processed reliably. 
In the context of databases, a single logical operation on the data is called a transaction. 

\subsubsection{Atomicity}

Atomicity requires that each transaction be "all or nothing": 
if one part of the transaction fails, the entire transaction fails, and the database state is left unchanged. An atomic system must guarantee atomicity in each 
and every situation, including power failures, errors, and crashes. To the outside world, a committed transaction appears (by its effects on the database) to be 
indivisible ("atomic"), and an aborted transaction does not happen.

\subsubsection{Consistency}

The consistency property ensures that any transaction will bring the database from one valid state to another. Any data written to the database must be valid 
according to all defined rules, including constraints, cascades, triggers, and any combination thereof. This does not guarantee correctness of the transaction 
in all ways the application programmer might have wanted (that is the responsibility of application-level code) but merely that any programming errors cannot 
result in the violation of any defined rules.

\subsubsection{Isolation}

The isolation property ensures that the concurrent execution of transactions results in a system state that would be obtained if transactions were 
executed serially. Providing isolation is the main goal of concurrency control. Depending on concurrency control method 
(i.e. if it uses strict - as opposed to relaxed - serializability), the effects of an incomplete transaction might not even be visible to another transaction.

\subsubsection{Durability}

Durability means that once a transaction has been committed, it will remain so, even in the event of power loss, crashes, or errors. In a relational database, 
for instance, once a group of SQL statements execute, the results need to be stored permanently (even if the database crashes immediately thereafter). 
To defend against power loss, transactions (or their effects) must be recorded in a non-volatile memory.

\section{PostgreSQL}

PostgreSQL \cite{11} is an open source object-relational database management system. It runs on all major operating systems, including Linux, UNIX (AIX, BSD, HP-UX, 
SGI IRIX, Mac OS X, Solaris, Tru64), and Windows. It is fully ACID compliant, has full support for foreign keys, joins, views, triggers, and stored procedures 
(in multiple languages). It includes most SQL:2008 data types, including INTEGER, NUMERIC, BOOLEAN, CHAR, VARCHAR, DATE, INTERVAL, and TIMESTAMP. It also supports 
storage of binary large objects, including pictures, sounds, or video. Moreover, it has native programming interfaces for C/C++, Java, .Net, Perl, Python, Ruby, Tcl, ODBC. 
It supports international character sets, multibyte character encodings, Unicode, and it is locale-aware for sorting, case-sensitivity, and formatting. 
In addition, it supports unlimited database size, rows and indexes per table, 32TB table size, 1.6TB row size and 1GB field size. PostgreSQL manages 
database access permissions using the concept of roles. A role can be thought of as either a database user, or a group of database users, depending on how 
the role is set up. Roles can own database objects (for example, tables) and can assign privileges on those objects to other roles to control 
who has access to which objects. Furthermore, it is possible to grant membership in a role to another role, thus allowing the member role use of privileges 
assigned to the role it is a member of. Finally, PostgreSQL comes with several extensions that add extra capabilities in its function and usage. 

\subsection{PostGIS}

PostGIS \cite{12} is a spatial database extender for PostgreSQL object-relational database. It adds support for geographic objects allowing location queries 
to be run in SQL. It defines new data types, functions, operators and indexes especially for geographic objects. 

\subsubsection{Geography type}


Ενδεικτικά, το PostGIS επιτρέπει την αποθήκευση στην PostgreSQL των γεωγραφικών συντεταγμένων ενός σημείου στο χάρτη. Οι συντεταγμένες αυτές αποτελούνται από ένα 
ζεύγος αριθμών, το γεωγραφικό πλάτος (latitude) και το γεωγραφικό μήκος (longitude). Το PostGIS δίνει την επιλογή αποθήκευσης των συντεταγμένων ως ένα γεωγραφικό 
τύπο δεδομένων (geography type). Επιπλέον, παρέχει μία πληθώρα συναρτήσεων για αυτό το γεωγραφικό τύπο δεδομένων, που καθιστούν δυνατό τον υπολογισμό των πραγματικών 
αποστάσεων μεταξύ γεωγραφικών σημείων. Με τον τρόπο αυτόν, μπορούν για παράδειγμα να γίνουν επερωτήσεις σχετικά με το πλήθος σημείων που βρίσκονται σε συγκεκριμένη 
απόσταση από ένα κεντρικό σημείο (συνάρτηση ST\_DWithin).























