\chapter{Γεννήτρια Χωροχρονικών Δεδομένων}

\section{Αποθήκευση Χωρικών Δεδομένων}

Η γεννήτρια χωροχρονικών δεδομένων χρησιμοποιεί πραγματικά σημεία ενδιαφέροντος στο χάρτη ως πηγή χωρικών δεδομένων, όπως αυτά εμφανίζονται σε διαδικτυακές υπηρεσίες 
συστημάτων προτάσεων (recommendation systems) σαν το TripAdvisor \cite{1}. Τα σημεία αυτά αναγνωρίζονται από τις γεωγραφικές συντεταγμένες τους καθώς και τη 
διεύθυνση τους στο χάρτη. Επίσης, συνοδεύονται από βαθμολογίες και κριτικές, οι οποίες έχουν γίνει από χρήστες της υπηρεσίας TripAdvisor. Το πλήθος των  
διαθέσιμων σημείων ενδιαφέροντος για τη δημιουργία της γεννήτριας χωροχρονικών δεδομένων ανέρχεται σε συνολικό μέγεθος 13GB. Για την αποθήκευσή τους χρησιμοποιήθηκε 
η σχεσιακή βάση δεδομένων PostgreSQL. Το σχήμα της βάσης περιλαμβάνει δύο πίνακες, έναν για τα σημεία ενδαφέροντος και τα χαρακτηριστικά τους και έναν για τις κριτικές και 
βαθμολογίες που έχουν γίνει στα σημεία αυτά. 

\begin{figure}[H]
  \centering
  \includegraphics[width=0.7\textwidth]{figures/schema.png}
  \caption{Σχήμα βάσης δεδομένων για τα χωρικά δεδομένα}
\end{figure}

Πιο αναλυτικά, ο πίνακας για τα σημεία ενδιαφέροντος περιέχει τα εξής γνωρίσματα:

\begin{itemize}
 \item poisId: αναγνωριστικός ακέραιος αριθμός του σημείου ενδιαφέροντος, κύριο κλειδί.
 \item location: οι γεωγραφικές συντεταγμένες του σημείου. Χρήση του γεωγραφικού τύπου δεδομένων ο οποίος υποστηρίζεται από την επέκταση PostGIS της PostgreSQL.
 \item title: η ονομασία του σημείου.
 \item address: η διεύθυνση του σημείου.
\end{itemize}

Ένα σημείο ενδιαφέροντος μπορεί να έχει πολλές κριτικές, γι'αυτό και οι δύο πίνακες συνδέονται με σχέση 1 προς πολλά. Ο πίνακας για τις κριτικές περιέχει τα εξής 
γνωρίσματα:

\begin{itemize}
 \item revId: αναγνωριστικός ακέραιος αριθμός του σημείου ενδιαφέροντος στο οποίο αντιστοιχεί η κριτική, εξωτερικό κλειδί.
 \item rating: βαθμολογία του σημείου σε κλίμακα ακεραίων 1 έως 5.
 \item reviewTitle: ο τίτλος της κριτικής για το σημείο.
 \item review: το κείμενο της κριτικής για το σημείο.
\end{itemize}

Μετά την αποθήκευση όλων των διαθέσιμων χωρικών δεδομένων αναθέτουμε ένα ευρετήριο τύπου B-tree στο γνώρισμα poisId του πίνακα pois και στο γνώρισμα revId του πίνακα 
reviews. Επίσης, δημιουργούμε ένα ευρετήριο τύπου GiST στο γνώρισμα location του πίνακα pois. Με τις δομές αυτές θα μπορεί να γίνει αποδοτικά η αναζήτηση 
εγγραφών ως προς τους αναγνωριστικούς τους αριθμούς αλλά και ως προς την τοποθεσία ενός σημείου ενδιαφέροντος.

\section{Παράμετροι Εισόδου Γεννήτριας}

Η γεννήτρια χωροχρονικών δεδομένων λαμβάνεις τις εξής παραμέτρους εισόδου:

\begin{itemize}
 \item userIdStart: Ο αναγνωριστικός αριθμός του πρώτου χρήστη για τον οποίο θα δημιουργηθούν ημερήσιες τροχιές επισκέψεων σε σημεία ενδιαφέροντος.
 \item userIdEnd: Αντίστοιχα, ο αναγνωριστικός αριθμός του τελευταίου χρήστη.
 \item chkNumMean: Η μέση τιμή για το πλήθος των ημερήσιων επισκέψεων όλων των χρηστών στα διάφορα σημεία ενδιαφέροντος, το οποίο θα ακολουθεί την κατανομή Gauss.
 \item chkNumStDev: Αντίστοιχα, η διασπορά του πλήθους ημερήσιων επισκέψεων.
 \item chkDurMean: Η μέση τιμή της διάρκειας κάθε επίσκεψης σε σημείο ενδιαφέροντος, η οποία θα ακολουθεί την κατανομή Gauss.
 \item chkDurStDev: Αντίστοιχα, η διασπορά της διάρκειας κάθε επίσκεψης.
 \item dist: Η μέγιστη ακτίνα σε μέτρα στην οποία θα μπορεί να κινείται ένας χρήστης από μία επίσκεψη σε ένα σημείο στο επόμενο.
 \item maxDist: Η μέγιστη ακτίνα σε μέτρα από την κατοικία του χρήστη στην οποία θα μπορεί να κινείται κάθε μέρα.
 \item startTime: Η ώρα την οποία θα αρχίσει η πρώτη επίσκεψη της ημέρας για κάθε μέρα και κάθε χρήστη.
 \item endTime: Αντίστοιχα, η ώρα την οποία θα αρχίσει η τελευταία επίσκεψη της ημέρας για κάθε μέρα και κάθε χρήστη.
 \item startDate: Η πρώτη ημέρα δημιουργίας ημερήσιων τροχιών για όλους τους χρήστες.
 \item endDate: Αντίστοιχα, η τελευταία ημέρα δημιουργίας ημερήσιων τροχιών για όλους τους χρήστες.
 \item outCheckIns: Το αρχείο εξόδου για τις ημερήσιες επισκέψεις όλων των χρηστών.
 \item outTraces: Το αρχείου εξόδου για τα συνολικά στίγματα δορυφόρου που αντιστοιχούν στις ημερήσιες επισκέψεις και τροχιές όλων των χρηστών.
 \item outMaps: Το αρχείο εξόδου για τους χάρτες που απεικονίζουν τις ημερήσιες τροχιές όλων των χρηστών. 
\end{itemize}

\subsection{Παραδοχές}

TODO: πλήθος και διάρκεια - Gauss









