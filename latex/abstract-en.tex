%%%  Abstract, in English

\begin{abstracten}%

Nowadays, in the era of Big Data, the amount of social media data, that is being produced daily, increases significantly. The storage and analysis of such data cannot 
be achieved any more with traditional means and methods. Consequently, social networking services resort in using distributed systems and techniques, in order to store 
and manage effectively the huge amount of the data they own. Usually, the evaluation of such services results through the ease of use and satisfactory performance. 
However, deep understanding of how data is stored and managed cannot be reached, due to the lack of access to these data. In this way, it is not possible to 
evaluate properly such social networking services. Therefore, goal of the current diploma thesis is to design and implement a generator of realistic spatio-temporal and 
textual data, that will be similar to real social media data. 

The generator uses as source of data, real points of interest and reviews for these points, extracted by a well-known travel service. Then, it creates 
realistic daily routes per user on the map, using the Google Directions API. These daily routes are available in the form of static maps, using the 
Google Static Maps API. Each daily route includes check-ins at points of interest, together with rating and review of the point, and gps traces 
indicating the route. Also, the generator functions with various input parameters, that differentiate the amount of data produced. For example, such parameters 
can be the number of users created, the time period in which daily routes will be produced per user and the number and duration of daily check-ins. 

The data produced by the generator were stored in a distributed system, using storage tools that big social media platforms use as well. Afterwards, 
basic queries over these data were implemented. The execution of these queries in association with the 
amount of data and number of nodes in the distributed system indicated the perfomance behavior of such distributed systems. Therefore, we were able to 
understand and evaluate the distributed means of storage that most of the modern social networking services use, in order to manage their huge amount of data. 



 \begin{keywordsen}
  Generator, Spatiotemporal data, Textual data, Big Data, Points of Interest, Daily routes, Google Directions API, Google Static Maps API, Distributed Systems
 \end{keywordsen}
\end{abstracten}
