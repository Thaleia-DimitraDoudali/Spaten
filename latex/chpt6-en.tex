\chapter{Conclusion}

\section{Summary}

In this diploma thesis we were able to create a generator of spatio-temporal and textual social media data. These data resemble real data found in well-known online 
social networking services. The daily execution of the generator for a significant period of time resulted in the assemblance of a Big Data dataset of users with daily routes and 
check-ins to points of interest. In this way, we were able to re-create realistic social media data that we couldn't have access to due to privacy restrictions imposed by 
current social networking services. With these data available, we were able to evaluate the distributed storage system that HBase provides and is used by a significant 
number of such social networking services. Therefore, we stored the created dataset into an HBase cluster and implemented several queries that are common to such 
services. We, then, created a workload with these types of queries in order to test the scalability of the HBase cluster. After executing the scalability testing 
we resulted that the system is scalable, therefore, HBase provides indeed good performance and storage for Big Data social media services. 

\section{Related Work}

There are already generators that produce social media data. One such generator is DATAGEN, which is a social network data generator, provided by LDBC-SNB 
(Linked Data Benchmark Council - Social Network Benchmark \cite{26}). The LDBC-SNB aims to provide a realistic workload, consisting of a social network and a set of queries. 
DATAGEN is a synthetic data generator responsible for generating datasets for the three LDBC-SNB workloads: the Interactive, the Business Intelligence and the Analytical. 
DATAGEN mimics features of those found in real social network. They create a synthetic dataset due to the fact that it is difficult to find data with all the 
scaling characteristics their benchmark requires and collecting real data can be expensive or simply not possible due to privacy concerns.

Also, there is LFR-Benchmark generator \cite{27} which provides the network structure of a social network, being able to mimic some of the properties of real 
social media. The LFR generator produces social network graphs with power-law degree distribution and with implanted communities within the network.


\section{Future Work}

There are several ways in which the current diploma thesis could be extended. As fas the generator is concerned, the produced data could be enhanced by also adding more 
textual data such as Facebook posts or Twitter tweets. Also, the generator could use a photo and video pool in order to simulate the upload of photos and videos, 
which is a basic functionality of most social networking services. As far as the generator execution is concerned, the generator could be instantiated with other 
values in order to create a more polymorphic dataset. Also, the usage of Google Maps API for Work enables 100000 directions requests per 24 hour period, which is 
significantly more than the limit of 2500 daily requests for the users of the free API. Therefore, the generator can be used more efficiently for users 
subscribed to the Google Work services. 

Moreover, regarding the implemented queries, there are many more that can be implemented. Also, there could be queries over the data of GPS traces which are 
created by the generator. In this way, the created workload can be more efficient into the scalability testing of the HBase cluster. Finally, other 
distributed storage and management systems could be evaluated in a similar way, such as Hadoop. 
