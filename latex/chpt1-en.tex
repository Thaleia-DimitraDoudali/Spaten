\setcounter{chapter}{0}

\chapter{Introduction}

\setcounter{chapter}{1}

\section{Motivation}

Nowadays, the amount of social networking data that is being produced and consumed daily is huge and it is constantly increasing.
Moreover, these data can be in the form of text, photography, video, time and location. For example, the online social networking service Twitter 
has 302 million active users per month and manages 500 million tweets per day, according to official statistics \cite {1}. 
The tweets include textual data and a timestamp (the time of publication) and may also contain
image, video, and location. Traditional data storage and processing techniques are no longer sufficient for the huge amount of polymorphic 
social media data. Consequently, modern social networking services
resort to distributed systems and data management tools. For example, Twitter \cite {2} used to 
store tweets in traditional databases. However, as time passed and the amount of tweets increased significantly, Twitter had several issues in 
reading and writing data in these databases. Thus, Twitter created and used distributed storage tools in order to manage it's data. 
These tools gave notable boost in the performance of the service.

A large number of online services now use distributed data storage systems. For example, large online social media platforms 
such as Twitter and Yahoo!, use the distributed database system HBase \cite {3}, so as to manage their data \cite {4}. 
Similarly, many online social networking services like Facebook and Twitter,
use the distributed data storage and management system Hadoop \cite {5} \cite {6}. 
Even though we can retrieve a list of social media platforms that use certain distributed data management tools, 
we cannot have access in the data that these platforms use, due to user's privacy restrictions. 
In this way, it is not possible to 
evaluate properly such social networking services.

In the current diploma thesis, we want to bridge the gap considering the full understanding of the function of social networking services.
Since we can not have access to the relevant data, we tried to create realistic social networking data, which are similar to real such data.
Hence, we designed and implemented a generator of realistic spatiotemporal and textual data, similar to 
those found in well-known social networking services such as Facebook.

\section{Thesis contribution}

\begin{enumerate}
  \item Design and implementation of an open source parameterized generator of spatio-temporal and textual social media data. 
  \item Creation of a large dataset of such complex realistic social media data using the generator.
  \item Insertion of the large dataset into a distributed storage system. 
  \item Evaluation of the storage system's performance and scalability.
\end{enumerate}

\section{Chapter outline}

In chapter 2 we present the theoretical background, which is important for someone to know, in order to understand concepts and terminologies used later on. 
More specifically, we analyze the tools used in the implementation of the generator, such as the database for the source data,
Google's services for the creation and presentation of user's routes and some mathematical concepts for the input parameters. Also,
we include thw distributed storage and data management tools, which were used in order to store the dataset created by the generator. 

In chapter 3 we describe in details the design of the generator.











