%%%  Abstract, in Greek

\begin{abstractgr}%

Η ποσότητα δεδομένων κοινωνικής δικτύωσης, η οποία παράγεται καθημερινά, αυξάνεται με ραγδαίους ρυθμούς. Η αποθήκευση και ανάλυση του τεράστιου αυτού όγκου δεδομένων 
δε μπορεί πλέον να πραγματοποιηθεί με παραδοσιακές τεχνικές. Σαν αποτέλεσμα, οι σύγχρονες υπηρεσίες κοινωνικής δικτύωσης καταφεύγουν σε κατανεμημένα συστήματα και εργαλεία, 
τα οποία τους παρέχουν επαρκή χώρο αποθήκευσης πληθώρας δεδομένων αλλά και μεθόδους ταχύτατης \linebreak επεξεργασίας τους. Η αξιολόγηση των υπηρεσίων αυτών μπορεί να γίνει 
μέσω της επίδοσής τους κατά τη διάρκεια χρήσης τους. Η πλήρης κατανόηση και αποτίμηση, όμως, της υποδομής και των τεχνικών αποθήκευσης και επεξεργασίας δεδομένων, τις 
οποίες ακολουθούν, δε μπορούν να πραγματοποιηθούν λόγω της αδυναμίας πρόσβασης στον όγκο δεδομένων τον οποίο διαχειρίζονται. Για το λόγο αυτό, σκοπός της 
παρούσας διπλωματικής εργασίας είναι η δημιουργία μίας γεννήτριας ρεαλιστικών χωροχρονικών δεδομένων μεγάλου όγκου, τα οποία θα προσομοιάζουν πραγματικά δεδομένα 
υπηρεσιών κοινωνικής δικτύωσης. 

Πιο συγκεκριμένα, η γεννήτρια διαθέτει ως πηγή δεδομένων πραγματικά σημεία ενδιαφέροντος και κριτικές για τα σημεία αυτά από γνωστή υπηρεσία κοινωνικής δικτύωσης. 
Στη συνέχεια, δημιουργεί ημερήσιες ρεαλιστικές τροχιές χρηστών στο χάρτη χρησιμοποιώντας την υπηρεσία εύρεσης διαδρομών της Google. Για κάθε τροχιά αποθηκεύει 
τα δορυφορικά στίγματα των \linebreak διαδρομών του χρήστη και τις επισκέψεις του στα σημεία ενδιαφέροντος, οι οποίες συνοδεύονται από βαθμολογία και κριτική του σημείου αυτού. 
Οι ημερήσιες αυτές τροχιές είναι διαθέσιμες σε μορφή στατικού χάρτη, όπως αυτός δημιουργείται από την αντίστοιχη υπηρεσία της Google. Η γεννήτρια λαμβάνει διάφορες 
παράμετρους εισόδου, οι οποίες διαφοροποιούν το συνολικά παραγόμενο όγκο δεδομένων. Ενδεικτικά, αυτές είναι το πλήθος χρηστών που θα δημιουργηθούν, το 
χρονικό διάστημα δημιουργίας ημερήσιων τροχιών καθώς και το πλήθος και η διάρκεια των ημερήσιων επισκέψεων. 

Τα παραγόμενα δεδομένα από τη γεννήτρια αποθηκεύτηκαν σε ένα κατανεμημένο σύστημα διαχείρισης δεδομένων μεγάλου όγκου, στα πρότυπα αυτών που χρησιμοιούν γνωστές 
πλατφόρμες κοινωνικής δικτύωσης, με σκοπό την αξιολόγηση των υπηρεσιών τις οποίες προσφέρουν. Η αξιολόγηση αυτή έγινε με την υλοποίηση βασικών επερωτήσεων στα δεδομένα. 
Ανάλογα το μέγεθος του όγκου των δεδομένων, το πλήθος των κόμβων του κατανεμημένου συστήματος και τη μεθοδολογία υλοποίησης των 
επερωτήσεων είναι εμφανής η διακύμανση της επίδοσης του συστήματος, η οποία και μας οδηγεί στη συνολική αποτίμηση των υπηρεσιών κοινωνικής δικτύωσης. 


  \begin{keywordsgr}
  Γεννήτρια, Χωροχρονικά δεδομένα, Δεδομένα μεγάλου όγκου, Σημεία ενδιαφέροντος, \linebreak Κατανεμημένα συστήματα, Υπηρεσίες Google
  \end{keywordsgr}
\end{abstractgr}
