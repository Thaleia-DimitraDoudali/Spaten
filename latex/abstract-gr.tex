%%%  Abstract, in Greek

\begin{abstractgr}%

Η ποσότητα δεδομένων κοινωνικής δικτύωσης, η οποία παράγεται καθημερινά, αυξάνεται με ραγδαίους ρυθμούς. Η αποθήκευση και ο διαμοιρασμός του τεράστιου αυτού όγκου δεδομένων 
δε μπορεί πλέον να πραγματοποιηθεί με παραδοσιακές τεχνικές. Σαν αποτέλεσμα, οι σύγχρονες υπηρεσίες κοινωνικής δικτύωσης χρησιμοιούν κατανεμημένα συστήματα 
διαχείρισης δεδομένων, 
τα οποία τους παρέχουν επαρκή χώρο αποθήκευσης πληθώρας δεδομένων αλλά και μεθόδους ταχύτατης επεξεργασίας τους. Η αξιολόγηση των υπηρεσίων αυτών μπορεί να γίνει 
μέσω της μελέτης της επίδοσής τους κατά τη διάρκεια χρήσης τους. Η πλήρης κατανόηση και αποτίμηση, όμως, της \linebreak υποδομής και των τεχνικών αποθήκευσης και επεξεργασίας δεδομένων, τις 
οποίες ακολουθούν, δε μπορούν να πραγματοποιηθούν λόγω της αδυναμίας πρόσβασης στον όγκο δεδομένων τον οποίο διαχειρίζονται. 
Αυτό οφείλεται στο γεγονός ότι πρόκειται για ιδιωτικά επιχειρισιακά δεδομένα που αφορούν πραγματικούς χρήστες και δεν μπορούν να 
εκμεταλευτούν ερευνητικά. Για το λόγο αυτό, σκοπός της 
παρούσας διπλωματικής εργασίας είναι η δημιουργία μίας γεννήτριας ρεαλιστικών \linebreak χωροχρονικών δεδομένων μεγάλου όγκου, τα οποία θα προσομοιάζουν πραγματικά δεδομένα 
\linebreak υπηρεσιών κοινωνικής δικτύωσης. 

Πιο συγκεκριμένα, η γεννήτρια διαθέτει ως πηγή δεδομένων πραγματικά σημεία ενδιαφέροντος και κριτικές για τα σημεία αυτά από γνωστή υπηρεσία κοινωνικής δικτύωσης. 
Στη συνέχεια, \linebreak δημιουργεί ημερήσιες ρεαλιστικές τροχιές χρηστών στο χάρτη χρησιμοποιώντας την υπηρεσία \linebreak εύρεσης διαδρομών της Google. Για κάθε τροχιά αποθηκεύει 
τα δορυφορικά στίγματα των διαδρομών του χρήστη και τις επισκέψεις του στα σημεία ενδιαφέροντος, οι οποίες συνοδεύονται από βαθμολογία και κριτική του σημείου αυτού. 
Οι ημερήσιες αυτές τροχιές είναι διαθέσιμες σε μορφή στατικού χάρτη, όπως αυτός δημιουργείται από την αντίστοιχη υπηρεσία της Google, 
καθώς επίσης και με τη μορφή raw data. Η γεννήτρια λαμβάνει διάφορες 
παράμετρους εισόδου, οι οποίες διαφοροποιούν το συνολικά παραγόμενο όγκο και τη μορφή των δεδομένων. Ενδεικτικά, αυτές είναι το πλήθος χρηστών που θα δημιουργηθούν, το 
χρονικό διάστημα δημιουργίας ημερήσιων τροχιών καθώς και το πλήθος και η διάρκεια των ημερήσιων επισκέψεων. 

Η ημερήσια εκτέλεση της γεννήτριας για ένα σημαντικό χρονικό διάστημα οδήγησε στη \linebreak δημιουργία ενός συνόλου χωροχρονικών δεδομένων και δεδομένων κειμένου μεγάλου όγκου. 
Πιο συγκεκριμένα, δημιουργήθηκαν 9464 χρήστες, 1586537 επισκέψεις χρηστών και 38800019 \linebreak δορυφορικά στίγματα, τα οποία αντιστοιχούν σε συνολικά δεδομένα μεγέθους 3 GB.
Το σύνολο των δεδομένων αυτών αποθηκεύτηκε σε μία κατανεμημένη βάση δεδομενων, στα πρότυπα αυτών που χρησιμοιούν γνωστές 
πλατφόρμες κοινωνικής δικτύωσης. Στη συνέχεια, υλοποιήθηκαν επερωτήσεις στα διαθέσιμα δεδομένα, οι οποίες είναι αντιπροσωπευτικές πραγματικών 
επερωτήσεων σε \linebreak αντίστοιχες πλατφόρμες. Τέλος, πραγματοποιήθηκε η εκτέλεση των επερωτήσεων αυτών στα \linebreak δεδομένα αυτά για μεταβλητό πλήθος 
κόμβων του κατανεμημένου συστήματος αποθήκευσης και μεταβλητό πλήθος ταυτόχρονων επερωτήσεων. Με τον τρόπο αυτό αξιολογήθηκε η κλιμακωσιμότητα του συστήματος αυτού, η οποία μας οδηγεί και σε μία 
ενδεικτική αποτίμηση των υπηρεσιών \linebreak κοινωνικής δικτύωσης, οι οποίες χρησιμοποιούν αντίστοιχα κατανεμημένα συστήματα αποθήκευσης και επεξεργασίας
δεδομένων μεγάλου όγκου.


  \begin{keywordsgr}
  Γεννήτρια, Χωροχρονικά δεδομένα, Δεδομένα μεγάλου όγκου, Σημεία ενδιαφέροντος, \linebreak Κατανεμημένα συστήματα, Υπηρεσίες Google, HBase, κλιμακωσιμότητα
  \end{keywordsgr}
\end{abstractgr}
