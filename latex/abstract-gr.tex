%%%  Abstract, in Greek

\begin{abstractgr}%

Η ποσότητα δεδομένων κοινωνικής δικτύωσης, η οποία παράγεται καθημερινά, αυξάνεται με ραγδαίους ρυθμούς. Η αποθήκευση και ανάλυση του τεράστιου αυτού όγκου δεδομένων 
δε μπορεί πλέον να πραγματοποιηθεί με παραδοσιακές τεχνικές. Σαν αποτέλεσμα, οι σύγχρονες υπηρεσίες κοινωνικής δικτύωσης καταφεύγουν σε κατανεμημένα συστήματα και εργαλεία, 
τα οποία τους παρέχουν επαρκή χώρο αποθήκευσης πληθώρας δεδομένων αλλά και μεθόδους ταχύτατης \linebreak επεξεργασίας τους. Η αξιολόγηση των υπηρεσίων αυτών μπορεί να γίνει 
μέσω της επίδοσής τους κατά τη διάρκεια χρήσης τους. Η πλήρης κατανόηση και αποτίμηση, όμως, της υποδομής και των τεχνικών αποθήκευσης και επεξεργασίας δεδομένων, τις 
οποίες ακολουθούν, δε μπορούν να πραγματοποιηθούν λόγω της αδυναμίας πρόσβασης στον όγκο δεδομένων τον οποίο διαχειρίζονται. Για το λόγο αυτό, σκοπός της 
παρούσας διπλωματικής εργασίας είναι η δημιουργία μίας γεννήτριας ρεαλιστικών χωροχρονικών δεδομένων μεγάλου όγκου, τα οποία θα προσομοιάζουν πραγματικά δεδομένα 
υπηρεσιών κοινωνικής δικτύωσης. 

Πιο συγκεκριμένα, η γεννήτρια διαθέτει ως πηγή δεδομένων πραγματικά σημεία ενδιαφέροντος και κριτικές για τα σημεία αυτά από γνωστή υπηρεσία κοινωνικής δικτύωσης. 
Στη συνέχεια, δημιουργεί ημερήσιες ρεαλιστικές τροχιές χρηστών στο χάρτη χρησιμοποιώντας την υπηρεσία εύρεσης διαδρομών της Google. Για κάθε τροχιά αποθηκεύει 
τα δορυφορικά στίγματα των \linebreak διαδρομών του χρήστη και τις επισκέψεις του στα σημεία ενδιαφέροντος, οι οποίες συνοδεύονται από βαθμολογία και κριτική του σημείου αυτού. 
Οι ημερήσιες αυτές τροχιές είναι διαθέσιμες σε μορφή στατικού χάρτη, όπως αυτός δημιουργείται από την αντίστοιχη υπηρεσία της Google. Η γεννήτρια λαμβάνει διάφορες 
παράμετρους εισόδου, οι οποίες διαφοροποιούν το συνολικά παραγόμενο όγκο δεδομένων. Ενδεικτικά, αυτές είναι το πλήθος χρηστών που θα δημιουργηθούν, το 
χρονικό διάστημα δημιουργίας ημερήσιων τροχιών καθώς και το πλήθος και η διάρκεια των ημερήσιων επισκέψεων. 

Η ημερήσια εκτέλεση της γεννήτριας για ένα σημαντικό χρονικό διάστημα οδήγησε στη δημιουργία ενός συνόλου χωροχρονικών δεδομένων και δεδομένων κειμένου μεγάλου όγκου. 
Το σύνολο των δεδομένων αυτών αποθηκεύτηκε σε μία κατανεμημένη βάση δεδομενων, στα πρότυπα αυτών που χρησιμοιούν γνωστές 
πλατφόρμες κοινωνικής δικτύωσης. Στη συνέχεια, υλοποιήθηκαν κάποιες επερωτήσεις στα διαθέσιμα δεδομένα, οι οποίες είναι αντιπροσωπευτικές πραγματικών 
επερωτήσεων σε αντίστοιχες πλατφόρμες. Τέλος, πραγματοποιήθηκε η εκτέλεση των επερωτήσεων αυτών στα δεδομένα αυτά για μεταβλητό πλήθος 
κόμβων του κατανεμημένου συστήματος αποθήκευσης και μεταβλητό πλήθος ταυτόχρονων επερωτήσεων. Με τον τρόπο αυτό αξιολογήθηκε η κλιμακωσιμότητα του συστήματος αυτού, η οποία μας οδηγεί και σε μία 
ενδεικτική αποτίμηση υπηρεσιών κοινωνικής δικτύωσης, οι οποίες χρησιμοποιούν αντίστοιχα κατανεμημένα συστήματα αποθήκευσης και επεξεργασίας 
δεδομένων μεγάλου όγκου.


  \begin{keywordsgr}
  Γεννήτρια, Χωροχρονικά δεδομένα, Δεδομένα μεγάλου όγκου, Σημεία ενδιαφέροντος, \linebreak Κατανεμημένα συστήματα, Υπηρεσίες Google, HBase, κλιμακωσιμότητα
  \end{keywordsgr}
\end{abstractgr}
