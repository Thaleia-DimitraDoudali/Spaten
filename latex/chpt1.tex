\chapter{Εισαγωγή}

\section{Κίνητρο}

Στη σύγχρονη εποχή, τα δεδομένα κοινωνικής δικτύωσης, τα οποία παράγονται και \linebreak 
διαχειρίζονται καθημερινά από τις αντίστοιχες υπηρεσίες, αυξάνονται με ραγδαίους ρυθμούς. 
Επιπλέον, τα δεδομένα αυτά περιέχουν πληροφορίες κείμενου, φωτογραφίας, βίντεο, χρόνου και τοποθεσίας. Ενδεικτικά, η διαδικτυακή υπηρεσία κοινωνικής δικτύωσης Twitter, 
σύμφωνα με τα επίσημα στατιστικά \cite{1}, διαθέτει 302 εκατομμύρια ενεργούς χρήστες κάθε μήνα και διαχειρίζεται κάθε μέρα 500 εκατομμύρια μηνύματα (tweets) 
τα οποία δημοσιεύονται από τους χρήστες. Τα tweets περιλαμβάνουν στοιχεία κειμένου και χρόνου (η ώρα δημοσίευσης) και μπορεί να περιέχουν επίσης 
εικόνα, βίντεο και τοποθεσία. Τα παραδοσιακά μέσα αποθήκευσης και τεχνικές επεξεργασίας δεν επαρκούν πλέον για τον τεράστιο όγκο των πολύμορφων 
κοινωνικών δεδομένων. Για το λόγο αυτό, οι σύγχρονες υπηρεσίες κοινωνικής δικτύωσης 
καταφεύγουν σε κατανεμημένα συστήματα και εργαλεία διαχείρισης δεδομένων. Για παράδειγμα, η υπηρεσία Twitter \cite{2} αρχικά 
αποθήκευε τα μηνύματα, γνωστά ως tweets, των χρηστών σε παραδοσιακές βάσεις δεδομένων. Όμως, με το πέρασμα του καιρού και την τεράστια αύξηση των 
tweets, υπήρχαν προβλήματα στην ανάγνωση και εγγραφή δεδομένων στις βάσεις αυτές. Έτσι, η υπηρεσία δημιούργησε και χρησιμοποίησε κατανεμημένα εργαλεία αποθήκευσης 
για τα δεδομένα τα οποία έπρεπε να διαχειριστεί. Με τα εργαλεία αυτά παρατηρήθηκε σημαντική βελτίωση στην επίδοση της υπηρεσίας. 

Ένας μεγάλος αριθμός από διαδικτυακές υπηρεσίες χρησιμοποιούν πλέον κατανεμημένα \linebreak συστήματα αποθήκευσης δεδομένων. Για παράδειγμα, μεγάλες διαδικτυακές 
υπηρεσίες, όπως το Twitter και το Yahoo!, χρησιμοποιούν το κατανεμημένο σύστημα βάσης δεδομένων HBase \cite{3}, για τη διαχείριση 
ενός τμήματος των δεδομένων τους \cite{4}. Αντίστοιχα, πολλές διαδικτυακές υπηρεσίες κοινωνικής δικτύωσης, όπως το Facebook και το Twitter, 
χρησιμοποιούν το \linebreak κατανεμημένο σύστημα διαχείρισης δεδομένων Hadoop \cite{5} \cite{6}. Η ελεύθερη γνώση των συστημάτων τα οποία χρησιμοποιούν οι 
υπηρεσίες αυτές μας δίνει μία πρώτη εικόνα της λειτουργίας τους. Όμως, τα δεδομένα τα οποία διαχειρίζονται δε γίνονται διαθέσιμα στο ευρύ κοινό, για λόγους 
διαφύλαξης των προσωπικών δεδομένων των χρηστών. Aν τα συστηματα που χρησιμοποιουν ειναι τα καταλληλα για τη διαχειριση τετοιων δεδομενων.


\section{Κύρια σημεία της εργασίας}

\section{Οργάνωση κειμένου}