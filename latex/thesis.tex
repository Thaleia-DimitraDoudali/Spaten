\documentclass[diploma]{softlab-thesis}


%%%
%%%  The document
%%%

\begin{document}

%%%  Various tables

\tableofcontents

%%%  Main part of the book

\mainmatter

\chapter{Εισαγωγή}

\section{Κίνητρο}


\chapter{Θεωρητικό Υπόβαθρο}

\section{Βάσεις δεδομένων}

Με τον όρο βάση δεδομένων εννοούμε μία συλλογή από οργανωμένα δεδομένα, τα οποία μπορούν διαχειριστούν με εύκολο τρόπο. Ένα σύστημα διαχείρισης βάσεων δεδομένων 
είναι ένα λογισμικό που αλληλεπιδρά με τον χρήστη και άλλες εφαρμογές, με σκοπό τον ορισμό, την πρόσβαση, τη διαχείριση και τις επερωτήσεις στα δεδομένα που είναι 
αποθηκευμένα στη βάση δεδομένων. Οι βάσεις δεδομένων χρησιμοποιούνται ευρέως σε πολλές υπηρεσίες που αφορούν διάφορους τομείς της ανθρώπινης δραστηριότητας, όπως 
υπηρεσίες που προσφέρονται από τράπεζες, πανεπιστήμια και τηλεπικοινωνίακούς φορείς.

\subsection{Σχεσιακή βάση δεδομένων}

Ο όρος σχεσιακή βάση δεδομένων χρησιμοποιείται για να περιγράψει μία βάση δεδομένων που οργανώνει τα δεδομένα της με γνώμονα το σχεσιακό μοντέλο δεδομένων. Σύμφωνα 
με το μοντέλο αυτό, τα δεδομένα οργανώνονται σε πίνακες από γραμμές και στήλες, με κάθε γραμμή να περιέχει ένα μοναδικό κλειδί. Ο πίνακας αντιπροσωπεύει μία οντότητα, 
κάθε γραμμή αντιστοιχεί σε μία εφαρμογή της οντότητας και οι στήλες αποτελούν χαρακτηριστικά της οντότητας. Ανάμεσα στους διάφορους πίνακες της βάσης μπορεί να υπάρχουν 
σχέσεις από 'έναν σε έναν', από 'έναν σε πολλούς' ή και από 'πολλούς σε πολλούς', όπως χαρακτηριστικά ονομάζονται. Πιο συγκεκριμένα, η σχέση είναι ένα σύνολο από 
εγγραφές-γραμμές πινάκων που έχουν κοινά γνωρίσματα-στήλες. Άλλοι όροι που συναντά κανείς σε μία σχεσιακή βάση δεδομένων είναι το πρωτεύον κλειδί, το οποίο είναι ένα 
μοναδικό αναγνωριστικό μίας εγγραφής σε έναν πίνακα. Όταν ένα πρωτεύον κλειδί, είναι κοινό και για έναν άλλον πίνακα, τότε γίνεται εξωτερικό κλειδί στον πίνακα αυτό. Τέλος, 
το ευρετήριο σε ένα γνώρισμα μίας σχέσης παρέχει γρηγορότερη πρόσβαση στα δεδομένα.

\subsection{Σύνολο ιδιοτήτων ACID}

Ένα σύστημα διαχείριση βάσεων δεδομένων πρέπει να είναι συμβατό με το σύνολο ιδιοτήτων ACID, έτσι ώστε η λειτουργία της βάσης να θεωρείται έγκυρη και αποτελεσματική. 
Πρόκειται για τέσσερις βασικές αρχές που εξασφαλίζουν την αξιόπιστη ολοκλήρωση των δοσοληψιών σε μία βάση δεδομένων. Οι αρχές αυτές είναι η ατομικότητα, η συνέπεια, 
η απομόνωση και η μονιμότητα. Πιο αναλυτικά, η ατομικότητα απαιτεί η τροποποίηση, που θα γίνει στη βάση δεδομένων, να πραγματοποιηθεί είτε ολόκληρη είτε καθόλου. 
Έτσι, αν ένα μέρος της δοσοληψίας αποτύχει, θα πρέπει να αποτύχει και ολόκληρη η δοσοληψία. Η συνέπεια διασφαλίζει ότι η βάση δεδομένων θα βρίσκεται πάντα σε συνεπή κατάσταση. 
Η απομόνωση προϋποθέτει ότι οι δοσοληψίες δε μπορούν να έχουν πρόσβαση σε δεδομένα της βάσης τα οποία τροποποιούνται εκείνη τη στιγμή από κάποια άλλη δοσολοψία. 
Τέλος, η μονιμότητα εγγυάται στον χρήστη ότι τα αποτελέσματα μίας ολοκληρωμένης δοσοληψίας θα παραμείνουν στη βάση δεδομένων ακόμη και σε περίπτωση σφάλματος ή 
διακοπής λειτουργίας του συστήματος διαχείρισης της βάσης δεδομένων. 


\section{PostgreSQL}

Η PostgreSQL είναι μία αντικειμενοστρεφής-σχεσιακή βάση δεδομένων ανοιχτού κώδικα. 
 

Η PostgreSQL παρέχει πλήρη υποστήριξη για εξωτερικά κλειδιά, συνενώσεις, όψεις και πληθώρα τύπων δεδομένων. Μπορεί να εγκατασταθεί στα περισσότερα δημοφιλή 
λειτουργικά συστήματα, όπως Unix και Windows. Επίσης, διαθέτει περιβάλλον ανάπτυξης κώδικα σε γλώσσες προγραμματισμού, όπως C/C++, Java, Perl, Python, Ruby. 
Μπορεί να διαχειριστεί εύκολα μεγάλους αριθμούς ταυτόχρονων χρηστών καθώς και μεγάλο όγκο δεδομένων. Ενδεικτικά, το μέγιστο μέγεθος πίνακα είναι 32ΤΒ, 
εγγραφής είναι 1.6ΤΒ, πεδίου 1GB, ενώ τα μέγιστα μεγέθη βάσης, πλήθους εγγραφών και ευρετηρίων σε κάθε πίνακα είναι απεριόριστα. Επιπλέον, δίνει τη δυνατότητα στον 
χρήστη να δημιουργήσει νέους τύπους δεδομένων αλλά και να ορίσει νέες συναρτήσεις. Επιπρόσθετα, διαχωρίζει τα δικαιώματα πρόσβασης στα σχήματα της βάσης με τη λογική 
των ρόλων. Κάθε ρόλος αντιπροσωπεύει έναν μεμονωμένο ή μία ομάδα χρηστών που έχουν με συγκεκριμένα δικαιώματα πρόσβασης στο κάθε σχήμα της βάσης. 

Η PostgreSQL διαθέτει στο χρήστη την επιλογή διάφορων ειδών ευρετηρίων, όπως B-tree, Hash, GiST. Ένα ευρετήριο επιτρέπει στο σύστημα της βάσης δεδομένων να βρίσκει 
και να ανακτά μία εγγραφή πολύ ταχύτερα από τη σειριακή αναζήτηση της εγγραφής στον πίνακα της βάσης. Η χρήση ενός ευρετηρίου δημιουργεί πιο σύνθετη αναζήτηση για το 
σύστημα της βάσης, αλλά αν χρησιμοποιηθεί σωστά μπορεί να βελτιώσει σημαντικά την επίδοση του συστήματος. Αν δεν καθοριστεί ρητά, η PostgreSQL χρησιμοποιεί B-tree 
ευρετήριο στο πεδίο του πίνακα στο οποίο αυτό ανατίθεται. Το Β-δέντρο είναι μία δενδρική δομή δεδομένων η οποία κρατά τα δεδομένα ταξινομημένα και επιτρέπει αναζητήσεις, 
σειριακή πρόσβαση, εισαγωγές και διαγραφές σε λογαριθμικό χρόνο. Είναι ουσιαστικά μία γενίκευση του δυαδικού δέντρου, όπου πλέον κάθε κόμβος μπορεί να έχει περισσότερα 
από δύο παιδιά. Το Β-δέντρο είναι ειδικά βελτιστοποιημένο για συστήματα που διαβάζουν και γράφουν μεγάλα μπλοκ δεδομένων.

%%%  End of document

\end{document}