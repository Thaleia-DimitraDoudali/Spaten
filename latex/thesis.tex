\documentclass[diploma]{softlab-thesis}


%%%
%%%  The document
%%%

\begin{document}

%%%  Various tables

\tableofcontents

%%%  Main part of the book

\mainmatter

\chapter{Εισαγωγή}

\section{Κίνητρο}


\chapter{Θεωρητικό Υπόβαθρο}

\section{PostgreSQL}

Η PostgreSQL είναι μία αντικειμενοστρεφής-σχεσιακή βάση δεδομένων ανοιχτού κώδικα. Είναι πλήρως συμβατή με τις τέσσερις βασικές αρχές που εξασφαλίζουν 
την αξιόπιστη ολοκλήρωση των δοσοληψιών στις βάσεις δεδομένων. Οι αρχές αυτές είναι η ατομικότητα, η συνέπεια, η απομόνωση και η μονιμότητα, γνωστές και ως σύνολο 
ιδιοτήτων ACID. Πιο αναλυτικά, η ατομικότητα απαιτεί η τροποποίηση, που θα γίνει στη βάση δεδομένων, να πραγματοποιηθεί είτε ολόκληρη είτε καθόλου. 
Έτσι, αν ένα μέρος της δοσοληψίας αποτύχει, θα πρέπει να αποτύχει και ολόκληρη. Η συνέπεια διασφαλίζει ότι η βάση δεδομένων θα βρίσκεται πάντα σε συνεπή κατάσταση. 
Η απομόνωση προϋποθέτει ότι οι δοσοληψίες δε μπορούν να έχουν πρόσβαση σε δεδομένα της βάσης τα οποία τροποποιούνται εκείνη τη στιγμή από κάποια άλλη δοσολοψία. 
Τέλος, η μονιμότητα εγγυάται στον χρήστη ότι τα αποτελέσματα μίας ολοκληρωμένης δοσοληψίας θα παραμείνουν στη βάση δεδομένων ακόμη και σε περίπτωση σφάλματος ή 
διακοπής λειτουργίας του συστήματος διαχείρισης της βάσης δεδομένων. 

Η PostgreSQL παρέχει πλήρη υποστήριξη για εξωτερικά κλειδιά, συνενώσεις, όψεις και πληθώρα τύπων δεδομένων. Μπορεί να εγκατασταθεί στα περισσότερα δημοφιλή 
λειτουργικά συστήματα, όπως Unix και Windows. Επίσης, διαθέτει περιβάλλον ανάπτυξης κώδικα σε γλώσσες προγραμματισμού, όπως C/C++, Java, Perl, Python, Ruby. 
Μπορεί να διαχειριστεί εύκολα μεγάλους αριθμούς ταυτόχρονων χρηστών καθώς και μεγάλο όγκο δεδομένων. Ενδεικτικά, το μέγιστο μέγεθος πίνακα είναι 32ΤΒ, 
εγγραφής είναι 1.6ΤΒ, πεδίου 1GB, ενώ τα μέγιστα μεγέθη βάσης, πλήθους εγγραφών και ευρετηρίων σε κάθε πίνακα είναι απεριόριστα. Επιπλέον, δίνει τη δυνατότητα στον 
χρήστη να δημιουργήσει νέους τύπους δεδομένων αλλά και να ορίσει νέες συναρτήσεις. Επιπρόσθετα, διαχωρίζει τα δικαιώματα πρόσβασης στα σχήματα της βάσης με τη λογική 
των ρόλων. Κάθε ρόλος αντιπροσωπεύει έναν μεμονωμένο ή μία ομάδα χρηστών που έχουν με συγκεκριμένα δικαιώματα πρόσβασης στο κάθε σχήμα της βάσης. 

Η PostgreSQL διαθέτει στο χρήστη την επιλογή διάφορων ειδών ευρετηρίων, όπως B-tree, Hash, GiST. Ένα ευρετήριο επιτρέπει στο σύστημα της βάσης δεδομένων να βρίσκει 
και να ανακτά μία εγγραφή πολύ ταχύτερα από τη σειριακή αναζήτηση της εγγραφής στον πίνακα της βάσης. Η χρήση ενός ευρετηρίου δημιουργεί πιο σύνθετη αναζήτηση για το 
σύστημα της βάσης, αλλά αν χρησιμοποιηθεί σωστά μπορεί να βελτιώσει σημαντικά την επίδοση του συστήματος. Αν δεν καθοριστεί ρητά, η PostgreSQL χρησιμοποιεί B-tree 
ευρετήριο στο πεδίο του πίνακα στο οποίο αυτό ανατίθεται. Το Β-δέντρο είναι μία δενδρική δομή δεδομένων η οποία κρατά τα δεδομένα ταξινομημένα και επιτρέπει αναζητήσεις, 
σειριακή πρόσβαση, εισαγωγές και διαγραφές σε λογαριθμικό χρόνο. Είναι ουσιαστικά μία γενίκευση του δυαδικού δέντρου, όπου πλέον κάθε κόμβος μπορεί να έχει περισσότερα 
από δύο παιδιά. Το Β-δέντρο είναι ειδικά βελτιστοποιημένο για συστήματα που διαβάζουν και γράφουν μεγάλα μπλοκ δεδομένων.

%%%  End of document

\end{document}