%%%  Abstract, in English

\begin{abstracten}%

Nowadays, in the era of Big Data, the amount of social media data, that is being produced daily, increases significantly. The storage and analysis of such data cannot 
be achieved any more with traditional means and methods. Consequently, social networking services resort in using distributed systems and techniques, in order to store 
and manage effectively the huge amount of the data they own. Usually, the evaluation of such services results through the ease of use and satisfactory performance. 
However, deep understanding of how data is stored and managed cannot be reached, due to the lack of access to these data which is imposed by 
privacy restrictions. In this way, it is not possible to 
evaluate properly such social networking services. Therefore, goal of the current diploma thesis is to design and implement a generator of realistic spatio-temporal and 
textual data, that will be similar to real social media data. 

The generator uses as source data, real points of interest and reviews for these points, extracted by a well-known travel service. Then, it creates 
realistic daily routes per user on the map, using the Google Directions API. These daily routes are available in the form of static maps, using the 
Google Static Maps API. Each daily route includes check-ins at points of interest, together with rating and review of the point, and gps traces 
indicating the route. Also, the generator functions with various input parameters, that differentiate the amount and structure of data produced. For example, such parameters 
can be the number of users created, the time period in which daily routes will be produced per user and the number and duration of daily check-ins. 

The generator was executed daily for a significant amount of time in order to create a Big Data dataset of spatio-temporal and textual data. 
More specifically, the generator created 9464 users, 1586537 check-ins and 38800019 GPS traces, which sum up to 3 GB data.
The dataset was stored in a distributed database system using a specific data storage model. Moreover, we implemented certain queries for these data, that are 
representative of queries imposed by the users of real social networking services. Finally, we created a workload of queries and executed them for 
different number of concurrent queries and different number of nodes of the distributed database system. In this way, we were able to perform a scalability 
testing to the system and evaluate the performance of distributed means of storage and processing of social media data used by many social networking services.

 \begin{keywordsen}
  Generator, Spatio-temporal data, Textual data, Big Data, Points of Interest, Daily routes, Google Directions API, Google Static Maps API, HBase, 
  Scalability Testing, Distributed Systems
 \end{keywordsen}
\end{abstracten}
